\documentclass[parskip=half, 
			   fontsize=11pt,
			   paper=a4]				
{scrartcl}
\usepackage{pdfpages} 
\usepackage[colorlinks=true,
			allcolors=blue, 
			bookmarks=false]
		{hyperref}

\usepackage[scaled]{helvet}
\renewcommand\familydefault{\sfdefault} 
\usepackage[T1]{fontenc}
\usepackage[autostyle=false, style=english]{csquotes}
\MakeOuterQuote{"}

\usepackage[fromalign=right,
				fromlogo=on, 
				foldmarks=off,  
				fromrule=afteraddress,
				subject=beforeopening,  % Placement of subject
				firstfoot=on]
			{scrletter}

\setkomavar{fromname}{\textbf{Devan Allen McGranahan}} % your name
\setkomavar{fromaddress}{
								Livestock \& Range\\
								Research Laboratory  \\
								Miles City, MT 59301 \\ 
								\href{mailto:Devan.McGranahan@usda.gov}{Devan.McGranahan@usda.gov}} 
\setkomavar{backaddress}{}
\setkomavar{signature}{
				Sincerely, \\
				{\includegraphics[width=5cm]{signature}} \\ 
				Devan Allen McGranahan \\ 
				Rangeland Research Ecologist }

\makeatletter
\@setplength{backaddrheight}{-1in}
\makeatother



\raggedright
\makeatletter
\setlength{\@tempskipa}{-3cm}%
\@addtoplength{toaddrheight}{\@tempskipa}
\makeatother

\makeatletter
\@setplength{refvpos}{5cm}
\makeatother

\areaset{6.5in}{10in}

\renewcommand{\raggedsignature}{\raggedright} % make the signature ragged right
\setkomavar{fromlogo}{{\includegraphics[width=0.5\textwidth]
	{logo} }}


% Footer 

 % contains scrletter

\newcommand{\journal}{\emph{Environmental Management}}


\vspace{-5em} 

\begin{document}
	\begin{letter}{Editor\\
		\journal }
\setlength{\parindent}{10pt}

\opening{Dear Sir or Madam,}  
		
Please consider the attached submission for publication in \journal. 

This submission is motivated by the positive experience my colleague and I recently enjoyed publishing in \journal~(DOI \href{https://doi.org/10.1007/s00267-022-01659-y}{10.1007/s00267-022-01659-y}). 
I have decided to move on with this work after a lack of response from the Editor in Chief at the journal to which we initially submitted, following mishandling by an Associate Editor. 
After our initial review, the paper was understandably rejected and a resubmission requested, due to the extent of the reviewer comments. 
Each were productive, and we responded completely (these responses are attached). 
However, despite no reviewer criticism on the resubmitted manuscript, the (presumably new?) AE assigned to the resubmission rejected the manuscript on the grounds that our approach to measuring two-dimensional rate of spread violated assumptions that are in fact limited to one-dimensional rate of spread. 
But recently, Dr. Mark Finney\textemdash almost certainly the leading wildland fire scientist in the US, if not the world\textemdash led a textbook in which his team described the superiority of 2-D measurements over 1-D measurements. 
We brought this to the EiC's attention in our appeal of the AE decision, but have gone several weeks without a response. 

As these data are of great interest to our stakeholders, we need to press on with publication. 
I think the scope and audience of \journal~is a good fit for us. 
We have explicitly referenced Finney et al's textbook into our Methods section to make this point clear. 
While I understand \journal~has no formal Portable Peer-Review process, I hope the editorial staff might take this context into account and consider an expedited review process. 

We look forward to working with the staff of \journal~to ensure our manuscript meets the standards of the journal. 

\vspace{-3em} 
\closing{} %eg. Regards
%\cc{another dude}
\end{letter}



\end{document}
