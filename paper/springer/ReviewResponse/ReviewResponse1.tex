\documentclass[parskip=half]{scrartcl}
\usepackage{booktabs}
\usepackage[margin=2.5cm]{geometry}
\usepackage[dvipsnames]{xcolor}

\title{\Large Weather and fuel as modulators of grassland fire behavior in the northern Great Plains}
\subtitle{Response to Review}
\author{ }
\date{}

% \pagenumbering{gobble}
\raggedright 

\newcommand{\AR}[1]
	{\color{PineGreen}AR: #1\color{black} \par }

\begin{document}

\maketitle

\vspace{-2cm} 

\AR{We appreciate the opportunity to revise our submission to \emph{Environmental Management}. 

Major points of revision include: 
\begin{itemize}
\item   
\item 
\item 

\end{itemize} 

We trust the editor will find these revisions address the important reviewer concerns. 
Author responses (AR:) to each reviewer comment follow inline below, as colored text. }

\section*{Associate Editor} 

The topic of this manuscript falls within the scope of the journal. Two reviewers have evaluated the manuscript - although they see potentially interesting results, they have also identified critical issues that need to be addressed. These issues are related to the limited availability of soil data, which is required to better understand the soil processes regulating the observed treatment effects, and to the excessive speculation, leading to claims not supported by the data. In light of their comments, which I support, I recommend major revisions.

\AR{We appreciate the review and thoughtful considerations from both editor and reviewers.} 


\section*{Reviewer 1}

The manuscript addresses fire behaviour in prescribed burns carried out in the northern Great Plains of the US. The authors concentrated their efforts on rate of spread and fire temperatures as measured by thermocouples and they analysed the variation as a function of weather-related variables. The work suffers from a number of shortcomings:
-	The futile measurement of temperatures, which are not really able to inform about fire behaviour or fire effects and thus do not have  correspondence with standard fire behavior characteristics;
-	Flame characteristics, which are valuable from the point of view of managers, were not monitored. Likewise, fireline intensity was not calculated (but it seems it could easily be derived, as the authors have RoS and fuel load – in this system nearly all of the elevated biomass is expected to burn, even if it’s not totally cured). However, FLI estimates would suffer from the same problems as RoS, see below;
-	Fuel moisture and fuel load do not distinguish between components;
-	Calculated RoS deviates from the standard, i.e. distinction between forward and backward spread rate, as it integrates variation in wind direction and, apparently, interacting flame fronts because of combined ignition patterns;
-	Fire behavior analyses are based on weather collected far from the fires

It’s interesting that the authors seem well aware of all these deficiencies. Were they aware of them before planning the work?  I understand that the study reflects a management setting. However, with all the above issues, I don’t see how it can contribute to improved prescribed burning management, let alone increased fire behaviour understanding in this system. 

\AR{We appreciate the reviewer's attention to detail and clear understanding of the wildland fire environment, and the challenges in properly characterizing it.  
The reviewer is correct in observing that we are well aware of the advantages and disadvantages of various approaches to measuring fire behavior. }

Specific comments

L8. Flame size is also very often used by fire scientists.

L13-L20. Important to mention that moisture-related atmospheric variables are important because they control dead fuel moisture content.

L22. To be accurate, energy release rate not always increase with available fuel. What always increase is the quantity of released energy.

L25. Although temperature varies within flame, all flames and all fires have the same temperature, from the tiniest candle flame to the highest-intensity crown fire. So, what potentially increases with time since fire is flame size and fireline intensity.

L30. Not sure what is meant with “outside the highly-cured context”. The effect of fuel moisture content is relevant throughout the entire range of other variables, e.g. wind speed.

L37. For decades fire ecologists have been wrong in this endeavour, and it was not for lack of warning, see for e.g. Van Wagner \& Methven in the 1970s. The value of temperatures measured during fires is very limited in relation to fire behavior or effects.

L79. It’s hard to reconcile the objectives of the study with the long (but not exhaustive) previous text explaining the problems with measuring ``fire temperatures''. Instead of temperatures why did the authors not measured flame length or height or calculated fireline intensity?

L162. Please provide information regarding the capability of the thermocouples to detect peak temperatures, namely the type of thermocouple and its diameter.

L191. So, fire weather records were systematically taken on-site or not? If they were, this should be described, and the results presented – weather stations are too far away to be useful to describe wind conditions during the fires.

L246-248. These results, like in many other studies, are very indicative of the problems associated with flame temperature measurement. Most of the values measured and reported are below the temperature of flaming combustion!

L259. The lack of association between RoS and weather does not surprise me, given that it was collected far away. Supplement Table 1 should be included with the main text and plots of RoS versus wind speed and other weather variables (and fuel moisture) could be included as supplementary material.

Fig. 3 and Table 1 should not appear before Discussion.

L274-277. This would be expected because fire spread in the elevated fuels and residual flaming or smouldering combustion in the ground is not expected owing to the nature of the fuel bed.

L360. This is clearly a relevant limitation of the RoS methodology and one that precludes characterizing the fires in terms of the benchmark (headfire RoS) or comparison with most experiments.

L384. So, why was not this done, as it is standard practice? And why was curing not assessed?



\section*{Reviewer 2}

This manuscript describes the relationship between fuel bed, weather, and three fire behavior metrics in low-intensity prescribed fires in two sites in North Dakota. The manuscript is generally clearly written, though I provide many suggestions for improvement below.

INTRODUCTION
First paragraph would benefit from clear statement of link between ``prescribed fire'' and ``wildland fire''. Also, the introduction reads like a textbook until the second-to-last paragraph. State in the first paragraph where you’re going with all this terminology – lack of information for the Northern Great Plains (NGP) and why it’s needed. Although the title and abstract indicate that this paper is about grasslands, some terms used in the Introduction would make more sense if its stated up front that all this information is about grassland wildland fire.

24: Greater fuel load is also attributable to vegetation composition, productivity of system (soil fertility, climate, etc.), time since/intensity of grazing, and a whole lot of other stuff.

30-32:  ``highly-cured context of wildfire seasons''?  See first comment – explanation of ``curing'' might be useful for some readers.

63-66:  Does rate of spread measure intensity or energy flux? If not, how does it make moot the third issue of temperature as a fire behavior metric? If so, make this connection clearer. This is critical given the emphasis on rate of spread in this manuscript.

76-78:  Do the papers cited in this paragraph test the effect of fuel beds on fire behavior in the NGP? Mention of that would be nice here, given the emphasis on fuel in the title, abstract, and Introduction.

METHODS

Consistency is needed in this section and throughout the manuscript in terms used for different scales of measurement. This becomes important in the results when it’s unclear what scale is being referenced. For example, in the first paragraph of the ``Data collection'' subsection, the 1-m triangles seem to be referenced as both ``plots'' and ``microplots''. Since all the data analysis and results seem to be using the 1-m data, stick with just one term. Unless there’s something different between plots and microplots.
163: Why 15 cm? What is the biological or physical significance of that height?
184, Figure 1, and elsewhere: Units need to be provided for VPD. 
214: Here or in the description of sites/fires, state how many years over which data were collected.
222-223: It would be nice for the reader to briefly remind them here exactly what the fuel and weather variables described above are.
227-229: The data (Figure 1) suggest that there was a greater difference in VPD than RH between the two sites. Justify (physically, biologically) using RH rather than VPD. 
RESULTS
Figure 1 caption: Add detail that Hettinger fires occurred in fall and Central grasslands fires occurred in spring.
Figure 2 caption:  Suggest changing to ``response variables in boxes'' (My eye barely registers the blue text compared to the light blue data points indicating CG vs. H). Caption also needs to explain what RH, DP, and VPD are.
235-237: Are you indicating ``drier air'' by higher RH? How is it different from ``more evaporative'' conditions? Please clarify in this paragraph and/or in methods when describing these weather variables.
246: Figure 1 indicates that only 25\% of the fires exceeded 325 degrees C. Clarify in Figure 1 caption the scale of the data presented are from, and clarify here what scale you are talking about when referencing ``fires''.
Table 1: Formatting went awry in t/df column.
DISCUSSION
288: ``will require collecting''
293-294: Confusing wording: Suggest changing to ``...surface temperatures were 110 deg C and 165 deg C for backfires and headfires, respectively...''
312-313, 318: Is this second Archibold study also 15-cm temperature, or surface? Also, use the same units for fuel load as your data (kg/m2) to make comparison to your results easier.
358-359: Figure 1 shows ~25\% of observations <60 deg C and <50\% <100 deg C. If I’m interpreting Figure 1 incorrectly, the caption of Figure 1 needs modifying to make interpretation easier.

\end{document}
